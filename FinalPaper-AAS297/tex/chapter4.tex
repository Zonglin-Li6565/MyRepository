Combining the content of the book I chose, and the response from the family
I interviewed, I can do some critical analysis of the influence of
self-employment on children caring.
\par
First, self-employment can really take away the time which supposed to spend
with children from parents. However, this is not necessary mean to be bad for
children. Granted, it could give a window to some kids with terrible
self-control to attempt society disapproved behavior or cause children got bad
school performance. The absence of parental control can directly count for that.
However, things doesn't always go as it looks. Whether a child can success
highly depends on his or her self-control ability \cite{article:SelfControl}.
So, for those ``good kids'' who have a strong self-control, whether or not their
parents are around doesn't affect their performances at school, but for those
with weak self-control, even though their parents are looking at them 24 hours,
they still can develop bad behavior.
\par
Second, the involvement of children to parents business can have both good and
bad effects, in my opinion. The good thing is that since children are sometimes
required to do the works involving interaction with unknown people like tend the
cash registers, do the ordering for the clients since they generally better at
English than their parents, helping their parents can be a window opened for
children to interact with society, especially those older than them. This could 
help them to accumulate the experience of interacting with society when young, 
when they grow up, it will be easy for them to adjust. However, helping parents 
can cause barriers between friends of similar age, because they spend their
spare  time in their parents' stores or restaurants.
\par
From the view of contemporary American Society, self-employment really helped
the new Korean immigrants to overcome the difficulties when adjusting to a
totally new environment. For those without good English skills, there is no way
more effective than self-employment to help them settle down in America.
However, for those second generations and third generations, they have good
English, their education achievement are accepted by American companies. So,
much less of them choose self-employment than their first generation ancestries
\cite{online:SecondGenerationSelfEmployment}.
